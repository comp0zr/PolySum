\documentclass[a4paper, 10pt]{article}
\usepackage{amsmath}
\usepackage{amssymb}
\usepackage[brazilian]{babel}
\usepackage{minted}
\usemintedstyle{friendly}

\title{Prova do M\'etodo PolySum para se obter Aproxima\c{c}\~oes polinomiais de somat\'orios de polin\^omios}

\author{phi}

\newcommand{\PSum}{\sum_{i=1}^{x}P(i)}

\newcommand{\QExpanded}{c_1x + c_2x^2 + \cdots + c_{n+1}x^n+1}

\newcommand{\aij}{a_{i,j}}
\newcommand{\bij}{b_{i,j}}



\newcommand{\aijDef}{%
\begin{align*}
  a_{i,j} = \begin{cases}
    (n)_{i-1} & \text{se $j = n + 2$}\\
    (j)_{i-1} & \text{se $i \leq j < n + 2$}\\
    0 & \text{se $i > j$}
  \end{cases}
\end{align*}
}

\newcommand{\AMatrix}{%
$$
\left(
\begin{array}{ccccccc}
     (1)_0 & (2)_0  & (3)_0  & \cdots & (n)_0     & (n+1)_0     & (n)_0     \\
       0   & (2)_1  & (3)_1  & \cdots & (n)_1     & (n+1)_1     & (n)_1     \\
       0   &   0    & (3)_2  & \cdots & (n)_2     & (n+1)_2     & (n)_2     \\
    \vdots & \vdots & \vdots & \ddots & \vdots    & \vdots      & \vdots    \\
       0   &   0    &   0    & \cdots & (n)_{n-1} & (n+1)_{n-1} & (n)_{n-1} \\
       0   &   0    &   0    & \cdots & 0         & (n+1)_n     & (n)_n     \\
\end{array}
\right)
$$
}

\newcommand{\lemmaOne}{%
\begin{align*}
  \frac{d^m}{dx^m}\Big[(x + c)^n\Big] = \begin{cases}
    (n)_m (x + c)^{n - m} & \text{se $m < n$} \\
    (n)_m & \text{se $m = n$} \\
    0 & \text{se $m > n$}
  \end{cases}
\end{align*}
}

\newcommand{\lemmaOneCaseOneBase}{%
\begin{align*}
  \frac{d^m}{dx^m}\Big[(x + c)^n\Big]
  &= \frac{d^0}{dx^0}\Big[(x + c)^n\Big] \\[1em]
  &= (x + c)^n \\[1em]
  &= 1 \cdot (x + c)^{n - 0} \\[1em]
  &= \frac{n!}{n!} \cdot (x + c)^{n - 0} \\[1em]
  &= \frac{n!}{(n - 0)!} \cdot (x + c)^{n - 0} \\[1em]
  &= (n)_0 (x + c)^{n - 0} \\[1em]
  &= (n)_m (x + c)^{n - m}
\end{align*}
}

\newcommand{\lemmaOneCaseOneInduction}{%
\begin{align*}
  \frac{d^m}{dx^m}\Big[(x + c)^n\Big]
  &= \frac{d^k}{dx^K}\Big[(x + c)^n\Big] \\[1em]
  &= \frac{d}{dx}\bigg[\frac{d^{k-1}}{dx^{k-1}}\Big[(x + c)\Big]\bigg] \\[1em]
  &= \frac{d}{dx}\Big[(n)_{k-1}(x + c)^{n - (k - 1)}\Big] \\[1em]
  &= (n)_{k-1}\frac{d}{dx}\Big[(x + c)^{n - (k - 1)}\Big] \\[1em]
  &= (n)_{k-1}\cdot (n - (k - 1)) \cdot (x + c)^{n - (k - 1) - 1} \\[1em]
  &= \frac{n!}{(n - (k - 1))!}\cdot (n - (k - 1)) \cdot (x + c)^{n - k + 1 - 1} \\[1em]
  &= \frac{n! \cdot (n - (k - 1))}{(n - (k - 1) - 1)! \cdot (n - (k - 1))} \cdot (x + c)^{n - k} \\[1em]
  &= \frac{n!}{(n - (k - 1) - 1)!} \cdot (x + c)^{n - k} \\[1em]
  &= \frac{n!}{(n - k + 1 - 1)!} \cdot (x + c)^{n - k} \\[1em]
  &= \frac{n!}{(n - k)!} \cdot (x + c)^{n - k} \\[1em]
  &= (n)_{k}(x + c)^{n - k} \\[1em]
  &= (n)_{m}(x + c)^{n - k}
\end{align*}
}

\newcommand{\lemmaOneCaseTwo}{%
\begin{align*}
  \frac{d^m}{dx^m}\Big[(x + c)^n\Big]
  &= \frac{d}{dx}\bigg[\frac{d^{m-1}}{dx^{m-1}}\Big[(x + c)^n\Big]\bigg] \\[1em]
  &= \frac{d}{dx}\bigg[\frac{d^{m-1}}{dx^{m-1}}\Big[(x + c)^n\Big]\bigg] \\[1em]
  &= \frac{d}{dx}\Big[(n)_{m-1}(x + c)^{n - (m - 1)}\Big] \\[1em]
  &= (n)_{m-1}\frac{d}{dx}\Big[(x + c)^{n - m + 1}\Big] \\[1em]
  &= (n)_{n-1}\frac{d}{dx}\Big[(x + c)^{n - n + 1}\Big] \\[1em]
  &= \frac{n!}{(n - (n - 1))!}\frac{d}{dx}\Big[(x + c)^{1}\Big] \\[1em]
  &= \frac{n!}{(n - n + 1)!}\frac{d}{dx}\Big[(x + c)^{1}\Big]  \\[1em]
  &= \frac{n!}{1!}\cdot\frac{d}{dx}\Big[x + c\Big] \\[1em]
  &= \frac{n!}{0!}\bigg(\frac{d}{dx}\Big[x\Big] + \frac{d}{dx}\Big[c\Big]\bigg) \\[1em]
  &= \frac{n!}{(n - n)!}(1 + 0) \\[1em]
  &= (n)_n \\[1em]
  &= (n)_m 
\end{align*}
}

\newcommand{\lemmaOneCaseThreeBase}{%
\begin{align*}
  \frac{d^m}{dx^m}\Big[(x + c)^n\Big]
  &= \frac{d}{dx}\bigg[\frac{d^{m-1}}{dx^{m-1}}\Big[(x + c)^n\Big]\bigg] \\[1em]
  &= \frac{d}{dx}\bigg[\frac{d^{n+1-1}}{dx^{n+1-1}}\Big[(x + c)^n\Big]\bigg] \\[1em]
  &= \frac{d}{dx}\bigg[\frac{d^n}{dx^n}\Big[(x + c)^n\Big]\bigg] \\[1em]
  &= \frac{d}{dx}\Big[(n)_n\Big]\\[1em]
  &= 0
\end{align*}
}

\newcommand{\lemmaOneCaseThreeInduction}{%
\begin{align*}
  \frac{d^m}{dx^m}\Big[(x + c)^n\Big]
  &= \frac{d^k}{dx^k}\Big[(x + c)^n\Big] \\[1em]
  &= \frac{d}{dx}\bigg[\frac{d^{k-1}}{dx^{k-1}}\Big[(x + c)^n\Big]\bigg] \\[1em]
  &= \frac{d}{dx}\Big[0\Big] \\[1em]
  &= 0
\end{align*}
}

\newcommand{\lemmaTwo}{%
\begin{align*}
  \sum_{i=1}^{x}P(i) &= Q(x) \\[1em]
  \sum_{i=1}^{x}P(i) - Q(x-1) &= Q(x) - Q(x-1)\\[1em]
  \sum_{i=1}^{x}P(i) - \sum_{i=1}^{x-1}P(i) &= Q(x) - Q(x-1)\\[1em]
  (P(x) + P(x-1) + \cdots + P(1)) - (P(x-1) + \cdots + P(1)) &= Q(x) - Q(x-1)\\[1em]
  P(x) + (P(x-1) + \cdots + P(1)) - (P(x-1) + \cdots + P(1)) &= Q(x) - Q(x-1)\\[1em]
  P(x) &= Q(x) - Q(x-1)
\end{align*}
}

\newcommand{\diffEquationSystem}{%
\begin{align*}
  \frac{d^0}{dx^0}\Big[Q(x) - Q(x-1)\Big] &= \frac{d^0}{dx^0}\Big[P(x)\Big] \\[1em]
  \frac{d^1}{dx^1}\Big[Q(x) - Q(x-1)\Big] &= \frac{d^1}{dx^1}\Big[P(x)\Big] \\[1em]
  \vdots \\[1em]
  \frac{d^n}{dx^n}\Big[Q(x) - Q(x-1)\Big] &= \frac{d^n}{dx^n}\Big[P(x)\Big] \\[1em]
\end{align*}
}

\newcommand{\ithEquation}{%
\begin{align*}
   \frac{d^{i-1}}{dx^{i-1}}\Big[Q(x) - Q(x-1)\Big] &= \frac{d^{i-1}}{dx^{i-1}}\Big[P(x)\Big] &\implies \\[1em]
  \frac{d^{i-1}}{dx^{i-1}}\Big[(c_1x + \cdots + c_{n+1}x^{n+1}) - (c_1(x - 1) + \cdots + c_{n+1}(x - 1)^{n+1})\Big] &= \frac{d^{i-1}}{dx^{i-1}}\Big[x^n\Big] &\implies \\[1em]
  \frac{d^{i-1}}{dx^{i-1}}\Big[c_1(x - (x - 1)) + \cdots + c_{n+1}(x^{n+1} - (x - 1)^{n+1})\Big] &= \frac{d^{i-1}}{dx^{i-1}}\Big[x^n\Big] &\implies \\[1em]
  c_1\cdot\frac{d^{i-1}}{dx^{i-1}}\Big[x - (x - 1)\Big] + \cdots + c_{n+1}\cdot\frac{d^{i-1}}{dx^{i-1}}\Big[x^{n+1} - (x - 1)^{n+1}\Big] &= \frac{d^{i-1}}{dx^{i-1}}\Big[x^n\Big] &\implies \\[1em]
\end{align*}
}

\newcommand{\unknowns}{c_1,\ldots,c_{n+1}}

\newcommand{\jthUnknown}{%
\frac{d^{i-1}}{dx^{i-1}}\Big[x^j - (x - 1)^j\Big]%
}

\newcommand{\jthUnknownSimp}{%
\bigg(\frac{d^{i-1}}{dx^{i-1}}\Big[x^j\Big] - \frac{d^{i-1}}{dx^{i-1}}\Big[(x - 1)^j\Big]\bigg)%
}

\newcommand{\independentValue}{%
\frac{d^{i-1}}{dx^{i-1}}\Big[x^n\Big]%
}

\newcommand{\caseOneComparison}{%
\begin{align*}
    j &= n + 2 &\implies\\
    j &> i &\implies\\
    j &> i - 1 &\implies\\
    i - 1 &< j 
\end{align*}
}

\newcommand{\caseOneResult}{%
\begin{align*}
a_{i,j}
&= \frac{d^{i-1}}{dx^{i-1}}\Big[x^n\Big] \\[1em]
&= (n)_{i-1} \\[1em]
\end{align*}
}

\newcommand{\caseTwoPartOne}{%
\begin{align*}
a_{i,j}
&= \frac{d^{i-1}}{dx^{i-1}}\Big[x^j\Big] - \frac{d^{i-1}}{dx^{i-1}}\Big[(x - 1)^j\Big] \\[1em]
&= (j)_{i-1}x^{j - (i - 1)} - (j)_{i-1}(x - 1)^{j - (i - 1)} \\[1em]
&= (j)_{i-1}\Big(x^{j - (i - 1)} - (x - 1)^{j - (i - 1)}\Big)\\[1em]
\end{align*}
}

\newcommand{\caseTwoPartTwo}{%
\begin{align*}
a_{i,j}
&= (j)_{i-1}\Big(x^{j - (i - 1)} - (x - 1)^{j - (i - 1)}\Big)\\[1em]
&= (j)_{i-1}\Big(1^{j - (i - 1)} - (1 - 1)^{j - (i - 1)}\Big)\\[1em]
&= (j)_{i-1}\Big(1^{j - (i - 1)} - 0^{j - (i - 1)}\Big)\\[1em]
&= (j)_{i-1}\Big(1 - 0\Big)\\[1em]
&= (j)_{i-1}\\[1em]
\end{align*}
}

\newcommand{\caseThree}{%
\begin{align*}
a_{i,j}
&= \frac{d^{i-1}}{dx^{i-1}}\Big[x^j\Big] - \frac{d^{i-1}}{dx^{i-1}}\Big[(x - 1)^j\Big] \\[1em]
&= (j)_{i-1} - (j)_{i-1} \\[1em]
&= 0\\[1em]
\end{align*}
}

\newcommand{\caseFour}{%
\begin{align*}
a_{i,j}
&= \frac{d^{i-1}}{dx^{i-1}}\Big[x^j\Big] - \frac{d^{i-1}}{dx^{i-1}}\Big[(x - 1)^j\Big] \\[1em]
&= 0 - 0 \\[1em]
&= 0\\[1em]
\end{align*}
}

\newcommand{\algorithmIthRow}{%
\begin{align*}
    a_{i,j} &= (j)_{i-2} \cdot (j - i + 2)\\[1em]
    a_{i,j} &= \frac{j!}{(j - (i-2))!} \cdot (j - i + 2)\\[1em]
    a_{i,j} &= \frac{j!\cdot (j - i + 2)}{(j - i + 2)!}\\[1em]
    a_{i,j} &= \frac{j!\cdot (j - i + 2)}{(j - i + 1)! \cdot (j - i + 2)}\\[1em]
    a_{i,j} &= \frac{j!}{(j - i + 1)!}\\[1em]
    a_{i,j} &= \frac{j!}{(j - (i - 1))!}\\[1em]
    a_{i,j} &= (j)_{i-1}
\end{align*}
}

\newcommand{\algorithmLast}{%
\begin{align*}
a_{n+1, n+2}
&= (n)_{n} \\[1em]
&= \frac{n!}{(n - n)!} \\[1em]
&= \frac{n!}{0!} \\[1em]
&= \frac{n!}{1!} \\[1em]
&= \frac{n!}{(n - (n - 1))!} \\[1em]
&= (n)_{n-1} \\[1em]
&= a_{n, n+2}
\end{align*}
}
\def\lang{portugues}

\begin{document}

\allowdisplaybreaks
\maketitle

\section{Teorema}
\subsection{Hipot\'ese}

Seja $P(x) = x^n$, $\PSum = Q(x) = \QExpanded$ e $A$ uma matriz $(n+1) \times (n+2)$ onde, sendo $\aij$ seu elemento na linha $i$ e coluna $j$

\aijDef

Ent\~ao se $B$ for a forma escalonada reduzida de $A$, onde $\bij$ \'e o elemento de $B$ na linha $i$ e coluna $j$, ent\~ao $c_i = b_{i,n+2}$.\\

Em outras palavras: Se contruir-mos a matriz (que podemos chamar de matriz base)

\AMatrix

Ent\~ao se aplicar-mos uma elimina\c{c}\~ao de Gauss-Jordan sobre esta matriz, ent\~ao o valor na ultima coluna da $i$-\'esima linha da matriz resultante ser\'a o coeficiente que multiplica o termo $x^i$ em Q(x).

\subsection{Lemas}
\subsubsection{Lema 1}

Para todos $n \geq 0$ e $m \geq 0$ onde $n$ e $m$ s\~ao n\'umeros naturais, e para todo $c$ independente de $x$, ent\~ao
\lemmaOne

\paragraph{Caso 1, $m < n$}

Provamos por indu\c{c}\~ao em $m$.

\subparagraph{Caso Base: $m = 0$}

\lemmaOneCaseOneBase

\subparagraph{Passo Indutivo: $m = k > 0$}

\lemmaOneCaseOneInduction
\paragraph{Caso 2: $m = n$}

\lemmaOneCaseTwo
\paragraph{Caso 3: $m > n$}
 
Provamos por indu\c{c}\~ao em $m$.

\subparagraph{Caso Base: $m = n + 1$}

\lemmaOneCaseThreeBase

\subparagraph{Passo Indutivo: $m = k > n + 1$}

\lemmaOneCaseThreeInduction
 
\pagebreak
\subsubsection{Lema 2}

Se $\PSum = Q(x)$, ent\~ao $P(x) = Q(x) - Q(x-1)$.

\paragraph{Demonstra\c{c}\~ao}

\lemmaTwo

\pagebreak

\subsection{Demonstra\c{c}\~ao do Teorema}

Seja $P(x) = x^n$ e $\PSum = Q(x) = \QExpanded$. Apartir do Lema 2, temos que $P(x) = Q(x) - Q(x-1)$, e podemos ent\~ao criar $n+1$ equa\c{c}\~oes tais que formemos o sistema

\diffEquationSystem

onde a $i$-\'esima equa\c{c}\~ao \'e

\ithEquation

Percebemos ent\~ao que estas s\~ao equa\c{c}\~oes lineares com incognitas $\unknowns$, onde o coeficiente de $c_j$ na equa\c{c}\~ao $i$ \'e $\jthUnknown$, ou $\jthUnknownSimp$, e o valor independente \'e $\frac{d^{i-1}}{dx^{i-1}}\Big[x^n\Big]$. Se montar-mos a matrix relacionada a este sistema ent\~ao obtemos uma matrix $A$ tal que, seja $a_{i,j}$ o seu elemento na linha $i$ e coluna $j$, se $j < n + 2$, ent\~ao $a_{i,j} = c_j$, e se $j = n+2$, $a_{i,j}$ \'e igual ao valor independente da equa\c{c}\~ao $i$.\\

Usamos agora o Lema 1 para simplificar essa matriz.

\pagebreak
\subsubsection{Caso 1: $j = n + 2$}

Como o maior valor poss\'ivel de $i$ \'e $n+1$, ent\~ao

\caseOneComparison

Logo:

\caseOneResult

\subsubsection{Caso 2: $i - 1 < j < n + 2$}

\caseTwoPartOne

Agora que eliminamos as deriva\c{c}\~oes, podemos substituir $x$ por $1$ sem perder generalidade:

\caseTwoPartTwo

\subsubsection{Caso 3: $i - 1 = j$}

\caseThree

\subsubsection{Caso 4: $i - 1 > j$}

\caseFour

Logo temos na verdade tr\^es casos, e como $i - 1 \geq j \implies i > j$, e $i - 1 < j \implies i \leq j$, ent\~ao:

\aijDef

Esta \'e exatamente a matriz base da hipot\'ese, e como ela \'e a matriz do sistema de equa\c{c}\~oes lineares cujas inc\'ognitas s\~ao os coeficientes de $Q$, ent\~ao de fato a ultima coluna da forma escalonada reduzida de $A$ nos dar\'a os valores destes coeficientes da maneira proposta pelo teorema.

\pagebreak
\section{Algoritmo}

\subsection{Hypothesis}
O seguinte algoritmo nos dar\'a a matriz base do m\'etodo:
\hrule
\inputminted[linenos]{code/\lang.py:AlgLexer -x}{code/\lang.alg}
\hrule

\subsection{Demonstra\c{c}\~ao}

Na primeira linha da matriz, a coluna ser\'a sempre maior ou igual \`a linha, logo 
$a_{1,j} = (j)_{1-1} = (j)_0$ para $1 \leq j \leq n+1$, e $a_{1,j} = (n)_{1-1} = (n)_0$
para $j = n+2$. Como para todo $k$, $(k)_0 = \frac{k!}{(k - 0 )!} = \frac{k!}{k!} = 1$,
ent\~ao todos os elementos na primeira linha s\~ao 1, e logo a primeira linha de $A$ est\'a
correta logo ap\'os a linha 3.

Agora para todo $i$ nas linhas de 6 at\'e 11, notamos que elas apenas modificam a linha $i$
de $A$, e como a primeira linha est\'a correta, podemos utilizar ``a linha anterior est\'a correta''
como nossa hipot\'ese que ser\'a provada se demonstrarmos que a $i$-\'esima linha tamb\'em
est\'a correta.

Nas linhas 7 at\'e 10 apenas as colunas de $i$ at\'e $n + 2$ s\~ao modificadas, ent\~ao para
$j < i$, $a_{i,j} = 0$, o que \'e correto.

Se $i \leq j < n + 2$, $a_{i,j} = a_{i-1, j} \cdot (j - i + 2)$. Como n\'os sabemos que que
todos os elementos na linha anterior est\~ao corretos, e como $i \leq j \implies i - 1 < j$,
$a_{i-1, j} = (j)_{i-2}$ e

\algorithmIthRow

O que \'e correto. A linha 10 do programa ser\'a correta se $a_{i,n+2} = a_{i,n}$, e como
$a_{i,n} = (n)_{i-1}$ para todo $i < n+1$, e $a_{i,n+2} = (n)_{n-1}$ para todo $i \leq n+1$,
ent\~ao isto \'e verdade para todas exceto a ultima linha. Como nenhuma outra linha depende
da ultima linha estar correta, n\~ao h\'a problema em deixa-la incorreta se a corrigirmos
mais tarde, o que ser\'a feito na linha 12 do programa. A linha 12 do programa \'e correta
pois
\algorithmLast

Assim a ultima linha da matriz estar\'a correta se a linha anterior o for, e esta estar\'a
correta se a linha anterior a ela o for, e assim por diante at\'e a linha 1, que j\'a provamos
estar correta. Logo, a matriz resultante $A$ est\'a completamente correta.

\end{document}
